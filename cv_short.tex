%------------------------------------
% Dario Taraborelli
% Typesetting your academic CV in LaTeX
%
% URL: http://nitens.org/taraborelli/cvtex
% DISCLAIMER: This template is provided for free and without any guarantee 
% that it will correctly compile on your system if you have a non-standard  
% configuration.
% Some rights reserved: http://creativecommons.org/licenses/by-sa/3.0/
%------------------------------------

%!TEX TS-program = xelatex
%!TEX encoding = UTF-8 Unicode

\documentclass[10pt, a4paper]{article}
\usepackage{fontspec} 

% DOCUMENT LAYOUT
\usepackage{geometry}
\geometry{a4paper, textwidth=5.5in, textheight=8.5in, marginparsep=7pt, marginparwidth=.6in}
\setlength\parindent{0in}

% FONTS
\usepackage[usenames,dvipsnames]{xcolor}
\usepackage{xunicode}
\usepackage{xltxtra}
\usepackage{url}

\defaultfontfeatures{Mapping=tex-text}
%\setromanfont [Ligatures={Common}, Numbers={OldStyle}, Variant=01]{Linux Libertine O}
%\setmonofont[Scale=0.8]{Monaco}
%%% modified by Karol Kozioł for ShareLaTeX use
%\setmainfont[
%  Ligatures={Common}, Numbers={OldStyle}, Variant=01,
%  BoldFont=LinLibertine_RB.otf,
%  ItalicFont=LinLibertine_RI.otf,
%  BoldItalicFont=LinLibertine_RBI.otf
%]{LinLibertine_R.otf}
%\setmonofont[Scale=0.8]{DejaVuSansMono.ttf}

% ---- CUSTOM COMMANDS
\chardef\&="E050
\newcommand{\html}[1]{\href{#1}{\scriptsize\textsc{[html]}}}
\newcommand{\pdf}[1]{\href{#1}{\scriptsize\textsc{[pdf]}}}

\usepackage{marginnote}
\newcommand{\amper{}}{\chardef\amper="E0BD }
\newcommand{\years}[1]{\marginnote{\scriptsize #1}}
\renewcommand*{\raggedleftmarginnote}{}
\setlength{\marginparsep}{7pt}
\reversemarginpar

% HEADINGS
\usepackage{sectsty} 
\usepackage[normalem]{ulem} 
\sectionfont{\mdseries\upshape\Large}
\subsectionfont{\mdseries\scshape\normalsize} 
\subsubsectionfont{\mdseries\upshape\large} 

% SKILL SECTION
\usepackage{tikz}
\definecolor{frontColor}{RGB}{47,79,79}% dark g
\definecolor{backColor}{RGB}{200,200,200}% grey
\newcommand{\grade}[1]{%
    \begin{tikzpicture}
    \clip (1em-.4em,-.35em) rectangle (5em +.5em ,1em);
    \foreach \x in {1,2,...,5}{
        \path[{fill=backColor}] (\x em,0) circle (.35em);
    }
    \begin{scope}
    \clip (1em-.4em,-.35em) rectangle (#1em +.5em ,1em);
    \foreach \x in {1,2,...,5}{
        \path[{fill=frontColor}] (\x em,0) circle (.35em);
    }
    \end{scope}

    \end{tikzpicture}%
}


% PDF SETUP
% ---- FILL IN HERE THE DOC TITLE AND AUTHOR
\usepackage[%dvipdfm, 
bookmarks, colorlinks, breaklinks, 
% ---- FILL IN HERE THE TITLE AND AUTHOR
	pdftitle={John Suárez-Pérez - vita},
	pdfauthor={John Suárez-Pérez},
	pdfproducer={http://nitens.org/taraborelli/cvtex}
]{hyperref}  
\usepackage[none]{hyphenat}
\hypersetup{linkcolor=blue,citecolor=blue,filecolor=black,urlcolor=MidnightBlue} 

% DOCUMENT
\begin{document}
{\LARGE John F. Suárez-Pérez}\\[1cm]
Ph.D. in Science - Physics (Expected: June 2023) \\
Bogotá, Colombia\\[.2cm]
Phone:\hspace{0.2cm}\texttt{ +57 3132843012 }\\
Email:\hspace{0.5cm}\href{mailto:jf.suarez@uniandes.edu.co}{jf.suarez@uniandes.edu.co}\\
Website:\hspace{0.16cm}\href{https://jsuarez314.gitlab.io}{https://jsuarez314.gitlab.io}\\
NASA/ADS:\hspace{0.16cm}\href{https://ui.adsabs.harvard.edu/search/p_=0&q=\%20author\%3A\%22Su\%C3\%A1rez-P\%C3\%A9rez\%2C\%20John\%20F.\%22&sort=date\%20desc\%2C\%20bibcode\%20desc}{ADS Abstract Service}

%%\hrule
%\section*{Current position}
%\emph{PhD Student}, Universidad de los Andes, Bogotá-Colombia,
%PhD Student of Physics in the Universidad de los Andes, searching on quantum spectroscopy and quantum optics.

%%\hrule
\section*{Research Interests}
\noindent
\textsc{Cosmology}\\
 Large-Scale Structure, Cosmic Web, Hydrodynamical Simulations.\\

\textsc{Extragalactic Astronomy}\\
Bright Galaxies, Luminous Red Galaxies, Galaxy Environment.\\

\textsc{Data Science}\\
Machine Learning, Graph Algorithms, Data Mining.

%\hrule
\section*{Formal Education}
\noindent
\years{2018}\textsc{M.Sc.} in Physics, Universidad de los Andes - Bogotá, Colombia, “Control experimental de las correlaciones en frecuencia de pares de fotones para una fuente de fotones individuales anunciados - Experimental Control of frequency correlations between photon pairs from an announced individual photons source".\\

\years{2014}\textsc{B.S.} in Physics, Universidad Distrital Francisco José de Caldas - Bogotá, Colombia, “Desarrollo de un aplicativo computacional para el estudio de circuitos con elementos memristivos - Computational software development for the study of circuits with memristive elements".\\


\section*{Further Education}
\noindent
\years{2021}\textsc{MOOC}, “Julia Scientific Programming". University of Cape Town. \href{https://coursera.org/share/faa0fcc877c44d0c38b6a2c4179818f2}{Coursera.}\\

\years{2020}\textsc{MOOC}, “Applied Machine Learning in Python". University of Michigan. \href{https://coursera.org/share/620fe326f4cc26b1f4a9bf77690114ae}{Coursera.}\\

\years{2019}\textsc{MOOC}, “Electrones en Acción: Electrónica y Arduinos para tus propios Inventos". Pontificia Universidad Católica de Chile. \href{https://www.coursera.org/account/accomplishments/certificate/W5M5CZ3XE557}{Coursera.}\\

\years{2019}\textsc{MOOC}, “Introducción a la programación en Python I: Aprendiendo a programar con Python". Pontificia Universidad Católica de Chile. \href{https://www.coursera.org/account/accomplishments/certificate/8UEF5TUY6ACB}{Coursera.}\\


\section*{Invited Research Visits}
\noindent
\years{2022}\textsc{Visiting Researcher}. Natural Science Research Institute, University of Seoul. Seoul-Republic of Korea. {August-October}.\\

\years{2022}\textsc{Visiting Researcher}. Intitute of Computational Cosmology, Durham University. Durham-UK. {June-July}.\\

\years{2019}\textsc{Visiting Researcher}. Max Planck Institute for Astrophyscis. Munich-Germany. {June}.\\

\section*{Honors and Awards}
\noindent
\years{2014}\textbf{Honour distinction for excellence in graduation work.} Laureate Thesis Award. Highest honour awarded for outstanding research work. Bachelor in Physics degree. Universidad Distrital Francisco José de Caldas - Bogotá, Colombia.

\section*{Publications, Events and Technologies}

\subsection*{Journal articles}
\noindent
\years{2021}\textbf{Suárez-Pérez, John F.} - Camargo, Yeimy D. - Xiao-Dong, Li; Forero-Romero, Jaime E., “The four cosmic tidal web elements from the $\beta$-skeleton", \emph{The Astrophysical Journal - ApJ}, \url{https://iopscience.iop.org/article/10.3847/1538-4357/ac1fed/pdf}
\\

\years{2020}Neira, Mauricio - Gómez, Catalina - \textbf{Suárez-Pérez, John F.} - Gómez, Diego A. - Reyes, Juan Pablo - Hernández Hoyos, Marcela - Arbeláez, Pablo - Forero-Romero, Jaime E., “MANTRA: A Machine Learning reference lightcurve dataset for astronomical transient event recognition", \emph{The Astrophysical Journal Series - APJs}, \url{https://iopscience.iop.org/article/10.3847/1538-4365/aba267/pdf} 
\\

\years{2020}\textbf{Suárez-Pérez, John} - Valencia, Alejandra - Nuñez, Mayerlin, “Characterization of spectrally filtered heralded single photons", \emph{Journal of the Optical Society of America B}, doi: \url{https://doi.org/10.1364/JOSAB.387118}

\subsection*{Events}
\noindent

\years{2022}\textbf{Suárez-Pérez, John} - Forero-Romero, Jaime E., DESI Team, “Assessing the quality of massive spectroscopic surveys with unsupervised machine learning" - \emph{XXXI IAUGA 2022},  Busan, Republic of Korea, Speaker.\\

\years{2022}\textbf{Suárez-Pérez, John} - Forero-Romero, Jaime E., DESI Team,  “Assessing the quality of massive spectroscopic surveys with unsupervised machine learning“ - \emph{ESA/ESO SCIOPS WORKSHOP 2022}, Remote - Max Planck Institute, Garching-Munich, Germany, Speaker.\\

\years{2021}\textbf{Suárez-Pérez, John} - Neira, Mauricio - Gómez, Catalina - Hernández Hoyos, Marcela - Arbeláez, Pablo - Forero-Romero, Jaime E., “TAO: Transient Astronomical Object Image Dataset" - \emph{Statistical Challenges in Modern Astronomy VII}, Remote - The Pennsylvania, Bogotá-Colombia, Speaker.\\
\newpage

\years{2020}\textbf{Suárez-Pérez, John} - Forero-Romero, Jaime - Camargo, Yeimy  Xiao, Dong-Li, “From the $\beta$-skeleton to cosmic web elements" - \emph{Latin American Workshop on Observational Cosmology}, Remote - ICTP-SAIFR, São Paulo-Brazil, Speaker.\\

\years{2020}\textbf{Suárez-Pérez, John} - Forero-Romero, Jaime - Camargo, Yeimy  Xiao, Dong-Li, “From the $\beta$-skeleton to cosmic web elements" - \emph{2nd CoCo meeting (Cosmología en Colombia) }, Remote - Universidad Antonio Nariño, Bogotá-Colombia , Speaker.\\

\years{2019}\textbf{Suárez-Pérez, John} - Forero-Romero, Jaime, “Machine Learning to reconstruct the dark matter density fields from galaxy survey" - \emph{XVI Latin American Regional IAU Meeting}, Universidad de Antofagasta, Antofagasta-Chile , Speaker.\\

\years{2019}\textbf{Suárez-Pérez, John} - Forero-Romero, Jaime, “Reconstructing the Universe with Machine Learning" - \emph{VI Congreso Colombiano de Astronomía y Astrofísica}, Parque Explora - Universidad de Antioquía, Medellín-Colombia , Speaker.\\

\years{2019}\textbf{Suárez-Pérez, John} - Forero-Romero, Jaime, “Understanding the large scale dark matter distribution with machine learning
algorithms"- \emph{1st CoCo meeting (Cosmología en Colombia)}, Universidad Antonio Nariño, Bogotá-Colombia, Speaker.\\

\subsection*{Technologies}
\noindent
\years{2014}\textbf{Suárez, John} - Castillo, Miguel - Salamanca, Julián, “Memcircuit", Software for the study of memristive circutis, \emph{Scientific Software}, 1-2014-64122.\\

\section*{Researching}
\noindent
\years{2019}\textsc{Research Assistant}. Universidad de los Andes, Trasients Project: Localize Transient Astronomical Objects on image sequences.\\

\section*{Teaching}
\noindent
\years{2018-2022}\textsc{Graduate Assistant Ph.D}. Universidad de los Andes. Computer Vision,  Computational Methods,  Basic Physics I.\\

\years{2016-2017}\textsc{Graduate Assistant M.Sc}. Universidad de los Andes. Experimental Physics I, Experimental Physics II, Basic Physics I Laboratory, Physics I,  Physics II,  Modern Optics.\\

\years{2015-2015}\textsc{Adjunct Professor}. Universidad de los Andes. Experimental Physics I,  Experimental Physics II.\\

\newpage
\section*{Outreach}
\noindent

\years{2021-2022}\textsc{Member}, \textit{ \href{https://www.astroreca.org/en/committeementors}{ Red de Estudiantes Colombianos de Astronomía - RECA}}  mentorship program committee.  A program focused on guiding students in Colombia who are interested in becoming professional astronomers.\\

\years{2020-2021}\textsc{Editor and Tutor}, \textit{\href{https://github.com/michaelJwilson/desihigh}{DESI High: School of the Dark Universe}} initiative, a project of DESI members focusing on introducing DESI to high schoolers through introductory notebooks written in python.

\section*{Skills and Abilities}

\subsection*{Software Development Skills}  
\begin{minipage}{0.5\textwidth}
\begin{tabular}{ll}
Python: & \grade{5}  \\
C++:& \grade{4.5}\\
\end{tabular}
\end{minipage}
\begin{minipage}{0.5\textwidth}
\begin{tabular}{ll}
Bash:& \grade{4}\\
Julia: & \grade{3}
\end{tabular}
\end{minipage}


\subsection*{Software Tools}  
\begin{minipage}{0.5\textwidth}
\begin{tabular}{ll}
Visualization: & \grade{5}  \\
\emph{matplotlib, gnuplot}\\
Data Analysis:& \grade{5}\\
\emph{numpy, scipy, pandas}\\
\end{tabular}
\end{minipage}
\begin{minipage}{0.5\textwidth}
\begin{tabular}{ll}
Machine Learning: & \grade{5}  \\
\emph{sklearn, pytorch}\\
Others: & \grade{3}\\
\emph{git, make}
\end{tabular}
\end{minipage}


\subsection*{Languages}  
\begin{minipage}{0.5\textwidth}
\begin{tabular}{ll}
Spanish: & \grade{5}  \\
English:& \grade{4}\\
\end{tabular}
\end{minipage}
\begin{minipage}{0.5\textwidth}
\begin{tabular}{ll}
Esperanto: & \grade{3}\\
\end{tabular}
\end{minipage}

%\hrule
%\section*{Service to the profession}

%\hrulefill
\vspace{2cm}
Scientists Prof. Jaime Forero-Romero (je.forero@uniandes.edu.co),  Prof. Shaun Cole\\ (shaun.cole@durham.ac.uk) and Prof. Cristiano Sabiu (csabiu@uos.ac.kr) have agreed to send reference letters for my application. Please do not hesitate to contact me should you require further information.
\vfill{}

\begin{center}
{\scriptsize  Last updated: \today\- •\- 
% ---- PLEASE LEAVE THIS BACKLINK FOR ATTRIBUTION AS PER CC-LICENSE
%Typeset in \href{http://nitens.org/taraborelli/cvtex}{
%\fontspec{Times New Roman}
%\XeTeX }\\
% ---- FILL IN THE FULL URL TO YOUR CV HERE
%\href{http://nitens.org/taraborelli/cvtex}{http://nitens.org/taraborelli/cvtex}
}
\end{center}
\end{document}