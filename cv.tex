%------------------------------------
% Dario Taraborelli
% Typesetting your academic CV in LaTeX
%
% URL: http://nitens.org/taraborelli/cvtex
% DISCLAIMER: This template is provided for free and without any guarantee 
% that it will correctly compile on your system if you have a non-standard  
% configuration.
% Some rights reserved: http://creativecommons.org/licenses/by-sa/3.0/
%------------------------------------

%!TEX TS-program = xelatex
%!TEX encoding = UTF-8 Unicode

\documentclass[10pt, a4paper]{article}
\usepackage{fontspec} 

% DOCUMENT LAYOUT
\usepackage{geometry} 
\geometry{a4paper, textwidth=5.5in, textheight=8.5in, marginparsep=7pt, marginparwidth=.6in}
\setlength\parindent{0in}

% FONTS
\usepackage[usenames,dvipsnames]{xcolor}
\usepackage{xunicode}
\usepackage{xltxtra}
\usepackage{url}

\defaultfontfeatures{Mapping=tex-text}
%\setromanfont [Ligatures={Common}, Numbers={OldStyle}, Variant=01]{Linux Libertine O}
%\setmonofont[Scale=0.8]{Monaco}
%%% modified by Karol Kozioł for ShareLaTeX use
%\setmainfont[
%  Ligatures={Common}, Numbers={OldStyle}, Variant=01,
%  BoldFont=LinLibertine_RB.otf,
%  ItalicFont=LinLibertine_RI.otf,
%  BoldItalicFont=LinLibertine_RBI.otf
%]{LinLibertine_R.otf}
%\setmonofont[Scale=0.8]{DejaVuSansMono.ttf}

% ---- CUSTOM COMMANDS
\chardef\&="E050
\newcommand{\html}[1]{\href{#1}{\scriptsize\textsc{[html]}}}
\newcommand{\pdf}[1]{\href{#1}{\scriptsize\textsc{[pdf]}}}

\usepackage{marginnote}
\newcommand{\amper{}}{\chardef\amper="E0BD }
\newcommand{\years}[1]{\marginnote{\scriptsize #1}}
\renewcommand*{\raggedleftmarginnote}{}
\setlength{\marginparsep}{7pt}
\reversemarginpar

% HEADINGS
\usepackage{sectsty} 
\usepackage[normalem]{ulem} 
\sectionfont{\mdseries\upshape\Large}
\subsectionfont{\mdseries\scshape\normalsize} 
\subsubsectionfont{\mdseries\upshape\large} 

% PDF SETUP
% ---- FILL IN HERE THE DOC TITLE AND AUTHOR
\usepackage[%dvipdfm, 
bookmarks, colorlinks, breaklinks, 
% ---- FILL IN HERE THE TITLE AND AUTHOR
	pdftitle={John Suárez-Pérez - vita},
	pdfauthor={John Suárez-Pérez},
	pdfproducer={http://nitens.org/taraborelli/cvtex}
]{hyperref}  
\hypersetup{linkcolor=blue,citecolor=blue,filecolor=black,urlcolor=MidnightBlue} 

% DOCUMENT
\begin{document}
{\LARGE John F. Suárez Pérez}\\[1cm]
Ph.D. Candidate of Science - Physics \\
Bogotá, Colombia\\[.2cm]
Phone:\hspace{0.2cm}\texttt{ +57 3132843012 }\\
Email:\hspace{0.5cm}\href{mailto:jf.suarez@uniandes.edu.co}{jf.suarez@uniandes.edu.co}\\
Website:\hspace{0.16cm}\href{https://jsuarez314.gitlab.io}{https://jsuarez314.gitlab.io}
%\textsc{url}: \href{http://www.ias.edu/spfeatures/einstein/}{http://www.ias.edu/spfeatures/einstein/}\\ 

%%\hrule
%\section*{Current position}
%\emph{PhD Student}, Universidad de los Andes, Bogotá-Colombia,
%PhD Student of Physics in the Universidad de los Andes, searching on quantum spectroscopy and quantum optics.

%%\hrule
\section*{Research Interests}
 Teaching of physics, computational cosmology, quantum optics, machine learning.

%\hrule
\section*{Formal Education}
\noindent
\years{2018}\textsc{M.Sc.} in Physics, “Control experimental de las correlaciones en frecuencia de pares de fotones para una fuente de fotones individuales anunciados", Universidad de los Andes - Bogotá, Colombia\\

\years{2014}\textsc{B.S.} in Physics,“Desarrollo de un aplicativo computacional para el estudio de circuitos con elementos memristivos", Universidad Distrital Francisco José de Caldas - Bogotá, Colombia\\


\section*{Further Education}
\noindent
\years{2020}\textsc{MOOC}, “Applied Machine Learning in Python". University of Michigan. \href{https://coursera.org/share/620fe326f4cc26b1f4a9bf77690114ae}{Coursera.}\\

\years{2019}\textsc{MOOC}, “Electrones en Acción: Electrónica y Arduinos para tus propios Inventos". Pontificia Universidad Católica de Chile. \href{https://www.coursera.org/account/accomplishments/certificate/W5M5CZ3XE557}{Coursera.}\\

\years{2019}\textsc{MOOC}, “Introducción a la programación en Python I: Aprendiendo a programar con Python". Pontificia Universidad Católica de Chile. \href{https://www.coursera.org/account/accomplishments/certificate/8UEF5TUY6ACB}{Coursera.}\\

\years{2019}\textsc{Visiting Researcher}. Max Planck Institute for Astrophyscis. Münich-Germany. {June 2019}\\

%\hrule
%\section*{Grants, honors \& awards}
%\noindent
%\years{1921}Nobel Prize in Physics, Nobel Foundation

\section*{Publications, Events and Technologies}

\subsection*{Journal articles}
\noindent
\years{2020}Neira, Mauricio - Gómez, Catalina - \textbf{Suárez-Pérez, John F.} - Gómez, Diego A. - Reyes, Juan Pablo - Hernández Hoyos, Marcela - Arbeláez, Pablo - Forero-Romero, Jaime E., “MANTRA: A Machine Learning reference lightcurve dataset for astronomical transient event recognition", \emph{The Astrophysical Journal Series - APJs}, arxiv: \url{https://arxiv.org/abs/2006.13163} 
\\
\newpage

\years{2020}\textbf{Suárez-Pérez, John} - Valencia, Alejandra - Nuñez, Mayerlin, “Characterization of spectrally filtered heralded single photons", \emph{Journal of the Optical Society of America B}, doi: \url{https://doi.org/10.1364/JOSAB.387118}
\\

\years{2018}\textbf{Suárez-Pérez, John} - Valencia, Alejandra - Nuñez, Mayerlin, “Control of the frequency correlations of entangled photons to produce a pure heralded single photon source", \emph{Latin America Optics and Photonics Conference Paper}, ISBN: 978-1-943580-49-1 \\

\years{2014}\textbf{Suárez, John} - Luengas, Danyela, “Diseño de una propuesta pedagógica para la enseñanza alternativa de las ciencias naturales", \emph{Revista Góndola}, Memorias Evento Académico, ISSN-E 2346-4712 \\

\subsection*{Events}
\noindent
\years{2020}\textbf{Suárez-Pérez, John} - Forero-Romero, Jaime, “Introducción a la análisis de datos del Dark Energy Spectroscopic Instrument (DESI)" - \emph{XXII Semana de la Enseñanza de la Física - SEF2020}, Virtual - Universidad Distrital Francisco José de Caldas, Bogotá-Colombia, Speaker.\\

\years{2020}\textbf{Suárez-Pérez, John} - Valencia, Alejandra - Nuñez, Mayerlin, “Characterization of spectrally filtered heralded single photons", \emph{QTurn 2020}, IQOQI Vienna.\\

\years{2020}\textbf{Suárez-Pérez, John} - Forero-Romero, Jaime - Camargo, Yeimy  Xiao, Dong-Li, “From the $\beta$-skeleton to cosmic web elements" - \emph{Latin American Workshop on Observational Cosmology}, Virtual - ICTP-SAIFR, São Paulo-Brazil, Speaker.\\

\years{2020}\textbf{Suárez-Pérez, John}, \emph{Encuentro Nacional de Enseñanza de la Ciencia para la Inclusión - ENECI}, Virtual - Universidad Distrital Francisco José de Caldas - Universidad de Pereira, Colombia , Assistant.\\

\years{2020}\textbf{Suárez-Pérez, John} - Forero-Romero, Jaime - Camargo, Yeimy  Xiao, Dong-Li, “From the $\beta$-skeleton to cosmic web elements" - \emph{2nd CoCo meeting (Cosmología en Colombia) }, Virtual - Universidad Antonio Nariño, Bogotá-Colombia , Speaker.\\

\years{2019}\textbf{Suárez-Pérez, John} - Forero-Romero, Jaime, “Machine Learning to reconstruct the dark matter density fields from galaxy survey" - \emph{XVI Latin American Regional IAU Meeting}, Hotel de Antofagasta - Universidad de Antofagasta, Antofagasta-Chile , Speaker.\\

\years{2019}\textbf{Suárez-Pérez, John} - Forero-Romero, Jaime, “Reconstructing the Universe with Machine Learning" - \emph{VI Congreso Colombiano de Astronomía y Astrofísica}, Parque Explora - Universidad de Antioquía, Medellín-Colombia , Speaker.\\

\years{2019}\textbf{Suárez-Pérez, John} - Forero-Romero, Jaime, “Reconstructing the Universe with Machine Learning" - \emph{X Escuela de Física Matemática: Machine learning for quantum matter and technology}, Universidad de los Andes, Bogotá-Colombia , Speaker.\\

\years{2019}\textbf{Suárez-Pérez, John} - Forero-Romero, Jaime, “Understanding the large scale dark matter distribution with machine learning
algorithms"- \emph{1st CoCo meeting (Cosmología en Colombia)}, Universidad Antonio Nariño, Bogotá-Colombia , Speaker.\\

\newpage

\years{2019}Pellaton, Matthieu - Villabona-Monsalve, Juan Pablo - \textbf{Suárez-Pérez, John} - Valencia, Alejandra - Nuñez-Portela, Mayerlin, “Measuring entangled two photon absorption cross sections and controlling the frequency correlations of paired photons for spectroscopic applications." \emph{26th Central European Workshop on Quantum Optics}, Paderborn University, Paderborn-Germany, Speaker.\\

\years{2018}\textbf{Suárez-Pérez, John} - Alvis, Elkin R.- Gonzaléz, Juan, González, María - Cuadrado, Marcela - Aguilera, Alejandra, Bermudez, Santiago - \emph{1st Joint Symposium in Optics: Topics on nonlinear phenomena}, Universidad Nacional - Universidad de los Andes, Bogotá-Colombia, Organizer.\\

\years{2018}\textbf{Suárez-Pérez, John} - Valencia, Alejandra - Nuñez, Mayerlin, “Experimental control of the frequency correlations for pure heralded single photons", \emph{IX Quantum Optics}, Cartagena-Colombia, Speaker.\\

\years{2017}\textbf{Suárez, John} - Nuñez, Mayerlin - Valencia, Alejandra, “Measurement of the heralded efficiency and the purity of a heralded single photons source", \emph{XXVII Congreso Nacional de Física} , Sociedad Colombiana de Física, Cartagena-Colombia, Speaker.\\

\years{2017}\textbf{Suárez, John} - Nuñez, Mayerlin - Valencia, Alejandra, “Experimental Control of Frequency Correlations of Entangled Photon Pairs", \emph{VI Quantum Information School and Workshop}, Sociedade Brasilera de Fisica, Paraty-Brasil, Speaker.\\

\years{2016}\textbf{Suárez, John} - Eraso, Leidy - Valencia, Alejandra, “Construcción de un perfilador láser portable mediante la adaptación de una webcam al microordenador Raspberry Pi", \emph{XIX Semana de la Enseñanza de la Física}, Universidad Distrital Francisco José de Caldas, Bogotá-Colombia, Speaker.\\

\years{2016}\textbf{Suárez, John} - Buesaquillo, Victor - Nuñez, Mayerlín, “Study of the spectral properties of entangled photons", \emph{Light and Matter School }, Universidad de los Andes, Bogotá-Colombia, Speaker.\\

\years{2014}\textbf{Suárez, John}, “Desarrollo y funcionamiento del Software Científico MEMCIRCUIT para el análisis de circuitos memristivos", \emph{XVII Semana de la Enseñanza de la Física }, Universidad Distrital Francisco José de Caldas, Bogotá-Colombia, Speaker.\\

\years{2014}\textbf{Suárez, John} - Luengas, Danyela, “Diseño de una propuesta pedagógica para la enseñanza alternativa de las ciencias naturales", \emph{XVII Semana de la Enseñanza de la Física }, Universidad Distrital Francisco José de Caldas, Bogotá-Colombia, Speaker.\\

\years{2013}\textbf{Suárez, John}, “Números pseudoaleatorios: Algoritmos de producción, evaluación de eficiencia, y aplicabilidad en el estudio de fenómenos físicos.", \emph{XVI Semana de la Enseñanza de la Física }, Universidad Distrital Francisco José de Caldas, Bogotá-Colombia, Speaker.\\

\years{2013}\textbf{Suárez, John}, “Estudio del problema de la aguja de Buffon para el cálculo de PI a través de un aplicativo computacional empleando la infraestructura de análisis de datos ROOT", \emph{XVI Semana de la Enseñanza de la Física }, Universidad Distrital Francisco José de Caldas, Bogotá-Colombia, Speaker.\\

\years{2013}Vargas, Andrés - \textbf{Suárez, John}, “Elementos conceptuales para abordar el uso de fotones polarizados para la distribución de claves criptográficas", \emph{XVI Semana de la Enseñanza de la Física }, Universidad Distrital Francisco José de Caldas, Bogotá-Colombia, Speaker.\\

\years{2013}\textbf{Suárez, John}, “Diseño e implementación de algoritmos en el paradigma de la programación orientada a objetos para la resolución de ODE's empleando la infraestructura de análisis de datos ROOT.", \emph{XVI Semana de la Enseñanza de la Física }, Universidad Distrital Francisco José de Caldas, Bogotá-Colombia, Speaker.\\

\years{2013}\textbf{Suárez, John}, “Desarrollo de un aplicativo informático para el análisis y la visualización del comportamiento de un circuito con elementos memristivos", \emph{XXV Congreso Nacional de Física}, Universidad del Quíndio, Armenia-Colombia, Speaker.\\

\years{2012}\textbf{Suárez, John}, \emph{XV Semana de la Enseñanza de la Física }, Bogotá-Colombia, Assistant.\\

\subsection*{Technologies}
\noindent
\years{2014}\textbf{Suárez, John} - Castillo, Miguel - Salamanca, Julián, “Memcircuit", Software for the study of memristive circutis, \emph{Scientific Software}\\

\section*{Researching}
\noindent
\years{2019}Research Assistant, Universidad de los Andes, Trasients Project: Localize Transient Astronomical Objects on image sequences.\\

\section*{Teaching}
\noindent
\years{2020}Graduate Assistant Ph.D., Universidad de los Andes, Computational Methods\\

\years{2019}Graduate Assistant Ph.D., Universidad de los Andes, Basic Physics II\\

\years{2018} Graduate Assistant M.Sc., Universidad de los Andes, Physics I, Physics II, Modern Optics\\

\years{2017} Graduate Assistant M.Sc., Universidad de los Andes, Experimental Physics I, Laboratory of Basics Physics I\\

\years{2016} Graduate Assistant M.Sc., Universidad de los Andes, Experimental Physics I, Experimental Physics II\\

\years{2015} University Professor, Universidad de los Andes, Experimental Physics I, Experimental Physics II\\

\years{2013} Science Teacher, Escuela Pedagógica Experimental (EPE)\\


%\hrule
%\section*{Service to the profession}
%\vspace{1cm}
\vfill{}
%\hrulefill

\begin{center}
{\scriptsize  Last updated: \today\- •\- 
% ---- PLEASE LEAVE THIS BACKLINK FOR ATTRIBUTION AS PER CC-LICENSE
%Typeset in \href{http://nitens.org/taraborelli/cvtex}{
%\fontspec{Times New Roman}
%\XeTeX }\\
% ---- FILL IN THE FULL URL TO YOUR CV HERE
%\href{http://nitens.org/taraborelli/cvtex}{http://nitens.org/taraborelli/cvtex}
}
\end{center}
\end{document}